% Created:  
% Author:   Julie Sewards
% Filename: introduction.tex

\chapter{Introduction}
\label{cha:introduction}
\section{Motivation}
\label{sec:motivation}
As the popularity of mobile communications devices increases, there is a growing tendency to use these as a convenient means of reviewing and revising documents on-the-move. Where these documents are of a confidential nature, particular attention must be paid to the fact that mobile devices are more vulnerable to compromise than traditional desktops, which are usually more extensively protected by the security measures implemented as part of an organization's internal network. 

There are multiple mechanisms for keeping files secure on company servers whilst allowing employees the necessary permissions to work collaboratively with sensitive data as required.  As the mobile device culture becomes more prevalent in the workplace, the addition of Mobile Device Management (MDM) applications empowers users to also access corporate data via their mobile devices whilst still allowing IT departments to retain a degree of control over data security.  

Thus, it is acknowledged that maintaining the security of confidential documents can be challenging, even with the weight of a corporate IT infrastructure behind it.  In this project, we seek to address the issue of allowing groups of users from different organizations (i.e. with no shared IT infrastructure) to collaborate securely on confidential documents and furthermore, to access these documents via a smartphone or tablet computer whilst minimizing the risk of exposing sensitive information to a potential attacker.


\section{Project Aims}
\label{sec:aims}
The primary aim of this project was to implement a scheme to facilitate secure sharing of confidential documents between a group of collaborators, subject to the following constraints:-
\begin{itemize}
\item Groups are self-organizing and represent multiple organizations, hence they cannot draw on the support of any central IT services.
\item The documents involved are confidential in nature and hence should be encrypted both in transit and at rest.
\item Group members wish to be able to access documents on a mobile device running the Android operating system (which may involve use of public wi-fi) without compromising security 
\item The solution devised should use only well-tested cryptographic techniques and standard libraries and should minimize the amount of trust to be placed in a third-party.
\end{itemize}

Our secondary objective was to gain an understanding of the basic features of the Android platform, explore some of the techniques involved in developing  for mobile devices and some of the challenges encountered in working in this type of event-driven environment.

In pursuit of these aims we developed a solution called SecurelyShare, consisting of a detailed design of the security components of the system and a prototype Android application to provide a platform on which to implement and evaluate the various security features.  It was acknowledged that, in a live setting, documents would usually originate on a PC rather than on a tablet device and thus the system would also need to a PC-based component.  However, within the time constraints of the project it was considered infeasible to develop a fully featured system; our solution is submitted rather as a 'proof of concept'.

\section{Overview of Report}
\label{sec:overview}

The subsequent chapters of this report will deal with the design, implementation and evaluation of the project.  Chapter \ref{cha:background} introduces some of the background material on key technologies used and presents an overview of the Android applications reviewed as part of our preliminary research.  In the light of this research, Chapter \ref{cha:analysis} presents a detailed analysis of the problem, defines the threat model against which we are attempting to defend, and provides an outline of the final solution design. Chapters \ref{cha:imp} and \ref{cha:mgt} provides details of the implementation, testing and management of the project and finally Chapter  \ref{cha:eval} evaluates the project components, presents a review of the overall success of project and gives recommendations for future work.