% Created:  
% Author:   Julie Sewards
% Filename: analysis-and-spec.tex

\chapter{Analysis and Design}
\label{cha:analysis}

In this chapter we will present some some of the salient points from our initial analysis of the problem, detail the threat model against which we are trying to protect and any significant assumptions made which influenced our design decisions, and finally outline our solution design.  It should be noted at this point that we had absolutely no prior experience of developing for (or even using) an Android device, therefore this phase was highly iterative.  As our familiarity and understanding of the capabilities of Android platform increased and as the design evolved, we were able to revisit these assumptions and the threat model in order to strengthen the security of the overall system. It is for this reason that we feel that presenting this material in a single chapter is more reflective of the underlying process.
\section{Overview}
At the outset, the following key areas were identified that would need further research:
\begin{itemize}
\item finalisation of the threat model and main assumptions
\item mechanism to be used for storage and distribution of shared documents
\item security design, including any key generation and distribution schemes
\item processes involved in setting up or managing groups
\item design of the prototype application, including any steps required to protect sensitive key material
\end{itemize}


\section{Problem Analysis}
\label{sec:prob}
\subsection*{Assumptions}

As outlined in the introduction, Securely Share is intended for use by small, autonomous groups (typically of the order of 3-15 members) with a largely static membership.  This factor was significant in enabling us to discard a number of complex key distribution protocols which were designed for much larger or more volatile groups.  It was also envisaged that groups would have lifetime measured in months, rather than days, hence a small administrative overhead in setting up the group could be considered acceptable in terms of the group duration. 

For practical purposes, it is likely that origination of sensitive documents would be carried out on a PC, as much for the practicality of typing as for any security reasons.  Therefore we are aware that one or more desktop client applications would be required for any fully-implemented solution.  For the purposes of this "proof of concept" project, it was decided that an outline version would be created in Java in order to verify the cryptography, but that time would not permit this to be developed and tested to a level where it could be submitted as part of the finished product.  

As discussed in section \ref{sec:crypto} the decision was made to use digital signatures as part of the security design. This generates the requirement for every group member to have a certificate that can be used for authentication. It is left to the discretion of each group to determine  whether this should be issued from a recognised Certificate Authority or whether self-signed certificates are to be allowed.  (For small groups, the process of manually confirming the certificate fingerprint by telephone may be acceptable, for example.)  In order to contain the scope of this project it was necessary to assume that each user either had, or was able to generate a public/private key pair and the associated certificate.  

Finally, it was assumed that the user acknowledges the sensitivity of his data and has taken reasonable steps to protect it, including the deployment of a device locking code and strong passwords.  

\subsection*{Threat and Trust Model}
For the purposes of this project, the following assumption regarding the threat and trust model were made:-
\begin{enumerate}[itemsep=3pt]
\item a group member may consider all other group members as trusted.
\item the Certificate Authority may be considered as trusted.  
\item all data in transit or at rest on a third-party server is considered to be exposed to an attacker.  In particular, as well  as a direct attack, data on an external server is subject to compromise by an insider within the third-party organisation or, as discussed in \ref{sec:other}, may be disclosed to comply with legal requests.
\item data at rest on the external storage of the device should always be considered as vulnerable to attack.  For example, an attacker may entice the user to install malware on the device(usually from an untrusted source) by presenting it as a new game, etc. - this can then be used to read data from the user's SD card.
\item  data at rest within the application's internal storage is protected to some extent by the platform's in-built security mechanisms, although this should still be encrypted to protect against a casual attacker who is able to gain root access for the purposes of snooping in storage.  The prototype is not required to implement software protection against a more sophisticated attacker who is able to exploit root access to implement key logging or to steal secrets from within the app memory.  However we are aware that the use of hardware-based cryptography via a Trusted Platform Module (TPM) would ameliorate the effect of this type of attack.
\end{enumerate} 


\section{Document Storage}
When considering how to manage the storage and sharing of group documents, we initially considered implementing a custom web server.  However, it was recognised that this option had a significant IT overhead, both in development and ongoing support and therefore did not fit with our requirements.  It was deemed that a better solution would be to leverage the facilities already available via one of the widely used cloud storage services:  Dropbox, Google Drive and OneDrive were considered.  

Although in a production-level application it would be desirable to allow the user to choose which cloud storage service they wished to use, it quickly became apparent that to implement using multiple APIs was infeasible within the scope of this project.  However, as the facilities offered by each were broadly similar, it was felt that developing for a single API in the prototype would be sufficient for the purposes of validating the concept.  Dropbox was selected as it is widely used and is well supported in both Windows and Linux environments, the latter being important as this was the development and testing environment being used for the project.  

As described in \ref{sec:dropbox}, Dropbox offers several version of the API for Android and the Sync API seemed to be the best match for the project requirements.  Developers are encouraged to assign each application the least permissions possible, so initially App level permission was chosen.  However, it came to light that this did not allow any access to shared folders so the design specification was amended to use the File type permission.  This presents a filtered view of the user's Dropbox folder, only permitting the calling application to see files of the specified types.  One disadvantage of this is that although all of the user's folders can be seen, various folder operations (e.g. delete, share) are prohibited as the folders may contain hidden files.  

When registering SecurelyShare, text and document types were required but it was also necessary to have a file extension to represent our encrypted files.  It would have been possible to submit a requrest to register custom file types with Dropbox but since time was of the essence, we elected to choose two less-commonly used ebook formats as a temporary measure, .xps (for encrypted documents) and .xeb (for encrypted group keys).  The .xps extension was simply added to the filename of an encrypted file so that it was easy to recover the original file extension on decryption.
\subsection*{Administration}
Although Dropbox provides an excellent environment for collaborative working with its inbuilt mechanisms for file and folder sharing, this does bring with it the initial administrative overhead of creating a group folder and sharing it with all the members.  As discusses, the Dropbox Sync API does not currently provide support for  this in Android but it was deemed acceptable for our purposes that this process should be completed via the Dropbox website in the usual way.  

\section{Cryptographic Design}
\label{sec:crypto}

There are two main aspects to the cryptographic design as follows:
\begin{itemize}
\item the document encryption scheme
\item the group key distribution protocol
\end{itemize} 
\subsection*{Document Encryption}
As one of the central assumptions is that any third-party server cannot be trusted, it is essential that client-side encryption is used to protect our sensitive data.  SecurelyShare uses a symmetric key encryption scheme and is currently implemented using the AES algorithm in Cipher Block Chaining (CBC) mode with a key length of 256 bits.  A design requirement was that references to algorithms and key lengths should be included as static variables rather than being hard-coded in order to provide an easy upgrade path or scope to make these user options at a later date.

Since a user of SecurelyShare may be a member of multiple groups at any one time, it is important that the decryption module is able to determine the correct key to use for each document.  Initially we thought to capture this information based on the name of the containing folder but quickly realised that this was prone to error as, if a file became disassociated from its correct folder there would be no way of determining where it came from and hence reuniting it with the correct key.  It was determined that each file should have a file header written to it made up of an integer containing the length (in bytes) of the group name, a string containing the group name and a fixed length Initialisation Vector - this would provide all the information required to decrypt the file.



\subsection*{Group Key Distribution}
By their nature, symmetric  schemes require that all parties are in possession of the key, as the same key is used for both encryption and decryption.  Where we are just sharing a key between two parties, methods such as the Diffie-Hellman  key exchange protocol \citep{diffie1976new} are widely used.  However, the problem of group key distribution is much more challenging and has been the subject of multiple research papers (see  \ref{sec:agke}).  Although these papers present a number of interesting theoretical schemes with advanced features, our design goal was to produce a practical but secure implementation which matches requirements built with well-established cryptographic primitives and protocols.    

Before generating the group key, the group administrator must set up a Java KeyStore containing the private key and corresponding certificate which is to be used to authenticate the group key on distribution.  This should be created on the PC to be used for running the Admin module and must be secured with a strong password which is then used by the KeyStore to encrypt the contents using PBE.  Each group member should supply a certificate which should then be imported into the KeyStore and finally the key generation module can be run.  

A high-level description of the key-generation process is as follows:-
\begin{itemize}
\item generate a new  symmetric key for the group
\item for each user certificate in the KeyStore, encrypt a copy of the group key with the user's public key (as contained in the certificate) using RSA
\item sign the encrypted key using the administrator's certificate
\item save the encrypted and signed key to the group Dropbox folder 
\end{itemize}

The use of digital signatures is required for each user to be able to validate that the encrypted key they receive is the authentic one and not one substituted by an attacker.  This  is a vital part of the  security protocol and hence has been included in our  system design.  However, as will be discussed in Chapter \ref{cha:imp}, due to some of the challenges encountered during the implementation phase, we were not able to implement this aspect of our design fully in the prototype.

\red{Should I discuss choice of PKI over ID based scheme here or in evaluation}\\

\red{Reasons why didn't choose ID based cryptography or password based solution - aim is to use simplest solution that works.   }\\

\section{Android Prototype }

\subsection*{Encryption }
As outlined in the Problem Analysis  (\ref{sec:prob}), it is unlikely that large documents will be originated from within SecurelyShare so our design just included a simple  edit window for text input which can then be saved to Dropbox as an encrypted file.  For completeness, and to aid testing, we also included the ability to encrypt a file which already exists on the device (although this is not to be encouraged as such files may have been exposed to an attacker prior to encryption).



\subsection*{Decryption }
Since tablet devices are ideally suited for reading documents, the major requirement of this app is that it is able to take encrypted documents stored in the user's Dropbox,  decrypt them with the appropriate symmetric key and present the original text to the user in a readable format.  Our requirements necessitated that this should be accomplished without the need to save the plaintext to disk on the device.  Initially this seemed as though it would be achievable as the Android platform provides mechanisms for one application to provide data to another,.  However, on deeper study, it transpired that none of these options was exactly right for our purpose.  We considered a number of ways of modifying our design in order to overcome this challenge  but all of them required some sort of modification to the third-party application.  Eventually it was decided that we had to weaken our requirements or risk jeopardising the entire project time scale. 

It was agreed that we could achieve our objective for simple text files by displaying the decrypted data in one of the Android-supplied Edittext widgets,which would enable it to be read and, if required, edited saved again in encrypted format.  For files requiring a more complex viewing and editing environment, for example those in Portable Document Format (PDF), we would write these temporarily to the external cache before broadcasting an Intent with the appropriate URI in order to allow the user to open this temporary file with one of the PDF readers installed on their device.  The security implications of this and suggestions for future work are discussed further in Chapter \ref{sec:security}.


\subsection*{Key Management }
When first installed on the Android device, the SecurelyShare application will create two KeyStores and save them in the application's private storage.  The Certificate keystore contains a copy of the user's private key and corresponding certificate chain and the Group keystore is the place where all the group keys will be stored.  

As described above, the group key is created by the group administrator and a file is generated for each group member containing a copy of the group key encrypted with the private key contained in the Certificate KeyStore.  Before the user can begin creating or reading documents for the group, this key must be imported into the Group keystore on the device.  The user selects his file from the Dropbox folder containing the encryption keys for the group, the app decrypts this with the user's private key and stores it in the Group keystore using the group name as an alias.  





