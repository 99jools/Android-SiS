% Created:  
% Author:   Julie Sewards
% Filename: analysis-and-spec.tex

\chapter{Analysis and Design}
\label{cha:analysis}

In this chapter we will present some some of the salient points from our initial analysis of the problem, detail the threat model against which we are trying to protect and any significant assumptions made which influenced our design decisions, and finally outline our solution design.  It should be noted at this point that we had absolutely no prior experience of developing for (or even using) an Android device, therefore this phase was highly iterative.  As our familiarity and understanding of the capabilities of Android platform increased and as the design evolved, we were able to revisit these assumptions and the threat model in order to strengthen the security of the overall system. It is for this reason that we feel that presenting this material in a single chapter is more reflective of the underlying process.
\section{Overview}
At the outset, the following key areas were identified that would need further research:
\begin{itemize}
\item finalisation of the threat model and main assumptions
\item mechanism to be used for storage and distribution of shared documents
\item security design, including any key generation and distribution schemes
\item processes involved in setting up or managing groups
\item design of the prototype application, including any steps required to protect sensitive key material
\end{itemize}


\section{Problem Analysis}
\label{sec:prob}
\subsection*{Assumptions}

As outlined in the introduction, Securely Share is intended for use by small, autonomous groups (typically of the order of 3-15 members) with a largely static membership.  This factor was significant in enabling us to discard a number of complex key distribution protocols which were designed for much larger or more volatile groups.  It was also envisaged that groups would have lifetime measured in months, rather than days, hence a small administrative overhead in setting up the group could be considered acceptable in terms of the group duration. 

For practical purposes, it is likely that origination of sensitive documents would be carried out on a PC, as much for the practicality of typing as for any security reasons.  Therefore we are aware that one or more desktop client applications would be required for any fully-implemented solution.  For the purposes of this "proof of concept" project, it was decided that an outline version would be created in Java in order to verify the cryptography, but that time would not permit this to be developed and tested to a level where it could be submitted as part of the finished product.  

As discussed in section \ref{sec:crypto} the decision was made to use digital signatures as part of the security design. This generates the requirement for every group member to have a certificate that can be used for authentication. It is left to the discretion of each group to determine  whether this should be issued from a recognised Certificate Authority or whether self-signed certificates are to be allowed.  (For small groups, the process of manually confirming the certificate fingerprint by telephone may be acceptable, for example.)  In order to contain the scope of this project it was necessary to assume that each user either had, or was able to generate a public/private key pair and the associated certificate.  

A final assumption was that the user acknowledges the sensitivity of his data and has taken reasonable steps to protect it, including the deployment of a device locking code and strong passwords.

\subsection*{Threat and Trust Model}
For the purposes of this project, the following assumption regarding the threat and trust model were made:-
\begin{itemize}
\item a group member may consider all other group members as trusted.
\item the Certificate Authority may be considered as trusted.  
\item all data in transit or at rest on a third-party server is considered to be exposed to an attacker.
\item the prototype Android application should aim to protect against an attacker accessing the device's external storage.  
\item  where an attacker is able to gain root access to the device, a casual attacker would then be able to access the application's protected internal storage and the prototype should take measures to protect against this.  The prototype is not required to implement software protection against a more sophisticated attacker with root access, however we are aware that the use of hardware-based cryptography via a Trusted Platform Module (TPM) would ameliorate the effect of this type of attack
\end{itemize}


\section{Document Storage}
When considering how to manage the storage and sharing of group documents, we initially considered implementing a custom web server.  However, it was recognised that this option had a significant IT overhead, both in development and ongoing support and therefore did not fit with our requirements.  It was deemed that a better solution would be to leverage the facilities already available via one of the widely used cloud storage services:  Dropbox, Google Drive and OneDrive were considered.  

Although in a production-level application it would be desirable to allow the user to choose which cloud storage service they wished to use, it quickly became apparent that to implement using multiple APIs was infeasible within the scope of this project.  However, as the facilities offered by each were broadly similar, it was felt that developing for a single API in the prototype would be sufficient for the purposes of validating the concept.  Dropbox was selected as it is widely used and is well supported in both Windows and Linux environments, the latter being important as this was the development and testing environment being used for the project.  









\section{Cryptographic Design}
\label{sec:crypto}

There are two main aspects to the cryptographic design as follows:
\begin{itemize}
\item the document encryption scheme
\item the key distribution protocol
\end{itemize} 

As one of the central assumptions is that any third-party server cannot be trusted, it is essential that client-side encryption is used to protect our sensitive data.  SecurelyShare uses a symmetric key encryption scheme and is currently implemented using the AES algorithm in Cipher Block Chaining (CBC) mode.

By their nature, symmetric  schemes require that all parties are in possession of the key, as the same key is used for both encryption and decryption.  Where we are just sharing a key between two parties, methods such as the Diffie-Hellman key exchange protocol \cite{dh1976}  are widely used.  However, the problem of group key distribution is much more challenging and has been the subject of multiple research papers (see  \ref{sec:agke})





\red{Reasons why didn't choose ID based cryptography or password based solution - aim is to use simplest solution that works.  Refer to examples which use PBE.  Happy to use it for protection of key material on device as even with password, attacker would have to have access to device or would require more sophisticated remote attack which is out of our scope.  }\\




\section{Group Administration}

Although Dropbox provides an excellent environment for collaborative working with its inbuilt mechanisms for file and folder sharing, this does bring with it the initial administrative overhead of creating a group folder and sharing it with all the members.  As the Dropbox API does not currently provide support for  this in Android, it was deemed acceptable that this process should be completed via the Dropbox website in the usual way.  

\red{include  screen shot of folder structure}\\

Before generating the group key, a Java KeyStore should be set up containing the private key and corresponding certificate which is to be used to authenticate the group key on distribution.  Each group member should supply a certificate which should then be imported into the KeyStore and finally the key generation module can be run.  

\red{figure should go here with a flowchart for this but will just include description for now}\\

A high-level description of the key-generation process is as follows:-
\begin{itemize}
\item generate a new  symmetric key for the group
\item for each user certificate in the KeyStore, encrypt a copy of the group key with the user's public key (as contained in the certificate)
\item sign the encrypted key using the administrator's certificate
\item save the encrypted and signed key to the group Dropbox folder 
\end{itemize}

The use of digital signatures is required for each user to be able to validate that the encrypted key they receive is the authentic one and not one substituted by an attacker.  This  is a vital part of the  security protocol and hence has been included in our  system design.  However, as will be discussed in Chapter \ref{cha:imp}, due to some of the challenges encountered during the implementation phase, we were not able to implement this aspect of our design fully in the prototype.


\section{Android Prototype }
\subsection*{Target Device}
\subsection*{Requirements}

Since tablet devices are ideally suited for reading documents, the major requirement of this app is that it is able to take encrypted documents stored in the user's Dropbox,  decrypt them with the appropriate symmetric key and present the original text to the user in a readable format.

As outlined in the Problem Analysis  (\ref{sec:prob}), it is unlikely that large documents will be originated from within SecurelyShare so our design just included a simple  edit window for text input which can then be saved to Dropbox as an encrypted file.  For completeness, and to aid testing, we also included the ability to encrypt a file which already exists on the device (although this is not to be encouraged as such files may have been exposed to an attacker prior to encryption).

\subsection*{Design}




