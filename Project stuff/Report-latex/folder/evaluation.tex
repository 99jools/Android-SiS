% Created:  23rd September 2014
% Author:   Julie Sewards
% Filename: introduction.tex

\chapter{Evaluation}
\label{cha:eval}
\section{Security Features}
\label{sec:security}
In designing our application we applied the principle of 'Defense in Depth' \citep{nsa}, using multiple layers of protection to guard against failure in any single mechanism.  Security features discussed earlier in this report include:
\begin{itemize}
\item linux and android inbuilt mechanism provide sandboxing and control access to internal file storage
\item the device should be locked with a pass code
\item the application has a master password that is used to encrypt the KeyStore containing all the cryptographic artefacts
\item data stored on both the device and Dropbox is encrypted 
\item encryption is performed on the device to ensure that the server has no access to decryption keys
\item using the Dropbox Sync API automatically ensures that all data in transit is protected using SSL
\item data on the server is also protected by the Dropbox access control facilities
\end{itemize}





\subsection*{Unauthorised Access to device}
In the event that the device is physically compromised (lost, stolen, etc.), we acknowledge that if an attacker is able to gain access to the device whilst the app is running he would be able to use the app to decrypt any documents in Dropbox as the KeyStore would be unlocked.  In the event that the SecurelyShare is not running, we have implemented a failed password lockout - encryption keys are deleted from the device after three successive password failures.  It should also be noted  that the use of Dropbox does provide some additional protection as, once aware that the device is lost, the user is able to log in via the Dropbox website and revoke SecurelyShare's permissions for the account with immediate effect. 

\subsection*{Password Policy}
In determining a password policy for SecurelyShare, we had to consider whether to require the user to enter the password for every file access, and whether to  use one password both to unlock the KeyStore and to encrypt each of the group keys or whether to have a separate password for each group.  It was felt that overuse of passwords can often lead to less security as users choose short passwords for speed of entry, or low entropy passwords to aid memory.  We feel that, with the added protections outlined above, the current approach is sufficient to meet requirements.


\subsection*{Key Distribution}

Each group requires an administrator to carry our the initial setup, although any member can serve in this role.  It is also possible to delegate this to someone who is not part of the group without giving them access to the group encryption key.   There is a recognised weakness in this solution if a corrupt administrator imports the attacker's certificate into the KeyStore, thus generating a copy of the encryption key for the attacker. After this point there would be no benefit for an attacker in subverting them, as being able to get access to the KeyStore containing all the certificates wouldn't give any advantage.  Gaining access to the Dropbox folder would enable the attacker to see what files were in there but without private keys he wouldn't be able to decrypt anything.


\subsubsection*{Backward Secrecy}
This is often required in a messaging environment where it is important that new group members are not able to decrypt messages from before they joined the group. However, in our application backward secrecy is not required as all group members should have access to historic documents.


\subsubsection*{Forward Secrecy and Key Revokation}
Requiring perfect forward secrecy adds an additional level of complexity to any encryption.   The main application of forward secrecy to our project is that a member who leaves the group should not have access to future documents.  Although our current cryptographic design does not provide this (once established, the group key is never changed), one of the benefits of using a service like Dropbox is that we are able to avail ourselves of its extensive access control mechanisms.  In order to compromise security an attacker needs access to both the group's shared Dropbox folder \textit{and } the group key - a member leaving the group would only have access to the former, which is considered sufficient for our purposes at this point.  Similarly, there is currently no means of revoking a key without the need to re-key all of the documents for the group (although again we could revoke access to the folder).   

This is one of the reasons why we suggest our solution my not be scalable to large groups as in this instance the membership is much more likely to be volatile.  An evaluation of more sophisticated key exchange protocols in order to strengthen this aspect of security without introducing dependency on a server for key management could be the subject of further work. 



\subsection*{Known Vulnerabilities}
As outlined in the design, to maintain full document security, we require that documents are only ever written to storage in encrypted format.  Although we were able to display and edit decrypted text files, decrypted PDF files have to temporarily be written to cache in order to pass them to the third-party application for reading and annotating.  This is felt to be an unsatisfactory solution but it was acknowledged that finding and customising a suitable PDF reader was not achievable within the time frame.  

 digital signatures to protect against man in the middle 
 
\section{Product}
scalability - Structure would work with larger group but there may be better key management protocols - admin issue with setting up dropbox shares etc.

PROTOTYPE EVALUATION
dependent on exactly correct alias for groupid and folder name
• Is designed as a “proof of concept”
• Aspires to use “best practice” within the code 
• Uses well-tested cryptographic techniques and standard libraries
• Adheres to the stated security requirements 
No ability to change passwords etc added at present
\section{Process}
major challenge of the fact that android is an operating system not a programming language - event driven programming

Since many of the example apps that we looked at were focused on personal encryption with the sharing capacity as an added feature, it was importatnt that we remembered that the core purpose of our application whas group collaboration and therefore central to any design we might chosose was the requirement for a robust key sharing mechanism.

development environment

\section{Overall Project}
EXPLAIN HOW WELL SOLUTION MEETS OBJECTIVES
ACHIEVEMENTS

WHAT YOU HAVE LEARNED

• Android is a whole new operating system not just ‘Java with extra bits’

Problem with API meant that we had to generate keystore for private key and corresponding certificate chain manually.  Further work would be to redesign this part of the app using the additional ffeatures offered in Android 4.3.  Currently consider this to be one of the weakest parts of the design - by the time I realised that this could have been done better, it was too late to go back and change it.

\section{Future Work}

look at generating keystores on device - would need to include method of exporting keystore (encrypted?) for distribution to other device.  

Review other schemes - lots of theoretical examples but anything that would work in practice

Implement proper user interface

Write a proper PC-based client application and look at improving group names

different levels of password policy as a user option - this may be a useful addition to be considered in the future.   Similarly, for ease of testing and general usability we decided not to implement encryption of filenames in the prototype,  However, this would again be a potential user option.


nelenkov.blogspot.co.uk - credential storage enhancements in Android 4.3












