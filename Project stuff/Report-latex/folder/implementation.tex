% Created:  
% Author:   Julie Sewards
% Filename: implementation.tex

\chapter{Implementation}
\label{cha:imp}
Implementation and testing. A detailed account of the implementation and testing of your software. Explain what data structures you used, and how the algorithms were implemented. What implementation decisions did you take, and why? There is no need to list every little function and procedure and explain its working in elaborate detail; use your judgement on what is appropriate to include. 
use of .xps, .xeb
For improvement, use custom file extension registered with Dropbox then would only ever see encrypted files
use of bundle for passing data between activities
use of interface for passing data back from dialog
Performance problem - introduced buffering
Splash screen and initialization
didn't use onStoreState etc. - didn't worry about restoring exact user position as prototype and system stores GUI stuff
Use of singleton
Keystores moved to external storage for testing and demonstration purposes.  In a production app, these should be moved back to internal storage in order to take advantage of the additional protection afforded by android’s inbuilt security mechanisms
Removal of keystores upon 3 successive failed password attempts – al present it just shows a message saying that the keystores have been deleted but doesn’t actually remove them from the device.  In a live system this would need to be implemented.
• use of xml rather than java for managing onClick - why was this done and when is it not applicable
Major issues with keystore, certificates and default providers.
Challenge of absence of built in file manager
DROPBOX ISSUES
• dbx stuff does not implement serialisable or parcelable
• Dropbox synchronization issues - developed everything using App specific access then discovered that this doesn't allow any use of shared folders so had to redesign 
Challenge of unavailability of BKS on pcs in school
No access to key tool in android
Include information about algorithms and key lengths
Fragments

User authentication and need to block after failed attempts
• large files
• network connectivity
• battery life
• small memory
• multithreading for gui
• where to encrypt
• model to use for file distribution and storage
TESTING
Testing - it is ok to say that I tested by inspection
Explain why unit testing is not meaningful

Issue: if generate [private key on device, it is device specific - ability to import would allow same keystores to be used on multiple devices for4 same user

Assumption that encrypted blobs are probably also created on PC - simple PC version of program developed to address this, although no gui developed

There were a number of design decision points where it was possible to do something in the prototype, it was possible that the user may want to do it infrequently but it was perfectly possible to do it from elsewhere, for example there are perfectly good mechanisms for creating and sharing Dbx folder  -and at present there is no means of sharing folders within the Sync API so there seemed little point in introducing the complexity of switching to a more complex API just for functionality that is easily done elsewhere and would only be required infrequently (it is really only a task required ast group setup and our view is that a group would be likely to have a lifetime of months

Use of Java program running on PC to develop encryption to begin with
No ability to recover keys if app gets deleted - could re download from Dropbox as long as private key and certificate backed up. 

3 parts

Admin system – basic with no front end.  Designed to run on system with Dropbox installed and running so files uploaded to dropbox simply by saving them in the correct location rather than worrying about using Dropbox API within program.

	Android prototype
	PC version in java that just does basic encryption/decryption.  Used primarily for intial development and testing of encryption mechanisms. Single username hardcoded for testing. Fully developed application with user interface outside of scope of this project.
	
	Android has inherent protection by sandboxing apps so files in internal storage have extra protection. FOr the purposes of development and testing, files written to app protected external storage so that they can be inspected with a file manager without the need to root the device

decision not to implement threads at this stage

When app is deleted, files should be deleted so keystore would be removed from device

I would develop prototype for a single API and test on one device – no backwards compatibility hence no need to use compatibility libraries eg for Action Bar

Write about choice of device

However, as will be discussed in Chapter \ref{cha:imp}, due to some of the challenges encountered during the implementation phase, we were not able to implement this aspect of our design fully in the prototype.

• how to manage letting decryption know group - flirted with shared preferences
Issue: if generate [rivate key on device, it is device specific - ability to import would allow same keystores to be used on multiple devices for4 same user