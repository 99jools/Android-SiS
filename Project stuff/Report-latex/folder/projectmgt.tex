% Created:  23rd September 2014
% Author:   Julie Sewards
% Filename: projectmgt.tex

\chapter{Project Management}
\label{cha:mgt}
We used an iterative approach, loosely based on Agile methodologies however with such a small development team and no external customer it was deemed inappropriate to use all of the porocesses and artifacts of any single agile methodology.  In particular although we strived to maintain working software at the end of each iteration this was not the production quality software which would be required in a true agile model where one is expected to deliver minimum viable product at the end of each iteration.

At the start of the project we knew very little about the android platform - we were starting from a zero-knowledge base and therefore more time was taken than anticipated in learning the android development environment, particularly with the proviso that in our initial aims we had set that we would seek to use best practice in protoype code.  During developent it becme apparent that timescales would have to be revised and this was duly done and well managed.

During development some elements which were required for security came to light.  Some of these were implemented in the prototype whereas others were merely noted in the security design.  There were parallel developments - the security design was improved over time as the knowledge base was increased. The other one was the prototype where on occasion we had to revise what was actually going to be achieveable within the time fram and some elements got moved from within the prototype into the design spec so that they became part of the security design but were not actually implemented.  For example the proviso of never writing plaintext to disc as is discussed in xxx was a requiremnt that had to be removed from the initial protoype deliverable.

Some significan challenges were encountered at the beginning to do with actually getting our development environment set up on the school computers.  This again impacted on the timescale as it exceeded the original margin allowed.




Version control was managed using a github repository.

No actual time was scheduled per task as there was no team of people to be organised.

One of the challenges of using an iterative development model was that since there was a lack of experience with the Android platform at the outset it was difficultto form a accurate pland and on occasion time was wasted on work that was critical to get a particular iteration to work but ultimately ended up being abandoned or didnt contribute significantly to the overall fiished product.  There was a danger of keeping things in the final design because of the amount of work that had actually gone into getting them working in the first place and this temptation was one that presented a challenge to overcome

Challenges of Agile methodology when using a new language or API - lot of time spent learning how to implement stuff that ended up not being needed.
Use of Java program running on PC to develop encryption to begin with
e.g. shared preferences - implemented during one of the iterations but ultimately abandoned for a simpler model
keystores may have been handled differently once had to use Bouncy Castle anyway

\url{http://www.scrumguides.org/scrum-guide.html#artifacts-increment}

One of the greatest challenges encountered was what is frequently described as "scope creep".  The role of the prototype as a proof of concept was clearly defined at the outset and therefore the design goal that functionality should take priority over any usability considerations.  