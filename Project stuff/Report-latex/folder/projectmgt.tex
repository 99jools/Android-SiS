% Created:  23rd September 2014
% Author:   Julie Sewards
% Filename: projectmgt.tex

\chapter{Project Management}
\label{cha:mgt}

In managing this project we used an iterative approach based on the Agile philosophy, although with such a small development team and no external customer many of the processes and artefacts associated with Agile methodologies were inappropriate.  In particular, although we had fully working software at the end of each iteration, it was questionable if it could have been considered a 'potentially shippable product' \citep{scrum}.

At the outset we had very little knowledge of the Android platform which resulted in it being difficult to form an accurate plan.  Although Android applications are generally programmed in Java with which we had some familiarity,  we had underestimated how much developing for  mobile devices differed from anything we had previously done.  We realised that a greater portion of the overall development time needed to be given over to production of the prototype, particularly because one of our initial aims was that we would seek to understand and implement best practice in our code.  As a result we had to readjust some of the timescales in our project plan, however this was recognised early and was able to be well managed.

One issue with using an iterative development model was that sometimes time was wasted on learning and development that was critical to get a particular iteration to work but which ultimately ended up being abandoned  or didn't contribute significantly to the final product.  It was important to guard against retaining features in the final design  simply because of the amount of work that had gone into implementing them in an earlier iteration.

During development some elements which were required for security came to light.  Some of these were implemented in the prototype, for example we deleted the KeyStore after  three failed password attempts, in order to protect against a device which had fallen into the wrong hands,  whereas others were merely noted in the security design.  Also, as discussed in Chapter \ref{cha:imp} we discovered a difficulty with opening decrypted PDFs without the need to write them out to a temporary file which made it necessary for us to revise the specification of the prototype. 

One of the greatest challenges encountered in managing the project was avoiding what is frequently described as "scope creep".  It was established at the outset that our development activities should be focussed on maximising the security of the prototype and yet we found it difficult to avoid getting sidetracked into adding features designed to enhance the user interface.  The process of delivering a regular progress report to our supervisor proved useful in helping to recognise and correct such digressions.


